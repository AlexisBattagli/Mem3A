\documentclass[12pt,a4paper]{report}
\usepackage[utf8]{inputenc}
\usepackage[T1]{fontenc}
\usepackage[french]{babel}
\usepackage[french]{nomencl}
\usepackage{lmodern}
\usepackage{graphicx} %Pour les images 
\usepackage{amsmath}
\usepackage{amsfonts}
\usepackage{amssymb}
\usepackage{makeidx}
\usepackage[left=2cm,right=2cm,top=2.5cm,bottom=2.5cm]{geometry}

%Interligne 1.5
\renewcommand{\baselinestretch}{1.5}

%Pour la page de garde
\newcommand{\hsp}{\hspace{20pt}}
\newcommand{\HRule}{\rule{\linewidth}{0.5mm}}

%HyperRef Conf
\usepackage{hyperref}
\hypersetup{
pdftitle={Mémoire de fin d'étude},
colorlinks=true, %colorise les liens
breaklinks=true, %permet le retour à la ligne dans les liens trop longs
urlcolor=black, %couleur des hyperliens
linkcolor=black, %couleur des liens internes
citecolor=black,    %couleur des liens de citations
bookmarksopen=true,
pdftoolbar=false,
pdfmenubar=false,
}

%Gestion des pied et en-tête de page
\usepackage{fancyhdr}
\pagestyle{fancy}
\lhead{}
\chead{}
\rhead{\leftmark}
\lfoot{Alexis BATTAGLI}
\cfoot{Page -\thepage-}
\rfoot{Mémoire de fin d'étude}
\renewcommand{\headrulewidth}{0.4pt}
\renewcommand{\footrulewidth}{0.4pt}

%Gestion des titre et indentation
\usepackage{titlesec}
\renewcommand{\thesection}{\arabic{section}}
\setcounter{secnumdepth}{4} % On affiche une numérotation sur une profondeur de 3
\setcounter{tocdepth}{4}        % La table des matières va a une profondeur de 3
% Alignement des titres :
\titlespacing{\chapter} {0pt} {*0} {*0} {} 
\titlespacing{\section} {4ex} {*0} {*0} {} 
\titlespacing{\subsection} {10ex} {*0} {*0} {} 
\titlespacing{\subsubsection} {18ex} {*0} {*0} {} 

%Gestion des Acronymes & Glossaire
\usepackage[acronym]{glossaries}
\makenoidxglossaries
\loadglsentries{MyGlossaries.tex}
\loadglsentries{MyAcronymes.tex}

\begin{document}

%Page de garde
\begin{titlepage}
  \begin{center}

    \textsc{\LARGE Mémoire de fin d'étude}\\[2cm]

    \textsc{\Large Ingénieur Informatique\\ spécialité Systèmes et Réseaux}\\[1.5cm]

    % Title
    \HRule \\[0.4cm]
    { \huge Conception et réalisation d'un outil de validation d'équipements CWMP\\[0.4cm] }

    \HRule \\[2cm]
    \includegraphics[scale=0.2]{./img/imt_mines_ales-bleu.jpg}
    \includegraphics[scale=0.1]{./img/orange.jpg}
    \\[2cm]

    % Author and supervisor
    \begin{minipage}{0.5\textwidth}
      \begin{flushleft} \large
        \emph{Alternant :} Alexis \textsc{BATTAGLI}\\
        \emph{Maitre d'apprentissage :}Marc \textsc{DOUET}\\
        \emph{Tuteur académique : } Yan \textsc{MORET}
      \end{flushleft}
    \end{minipage}
    \begin{minipage}{0.4\textwidth}
      \begin{flushright} \large
      	\emph{École :} IMT Mines Alès\\
       	\emph{Entreprise :} Orange\\
        \emph{Promotion :} INFRES 7\\
      \end{flushright}
    \end{minipage}

    \vfill

    % Bottom of the page
    {\large Septembre 2014 — Septembre 2017}

  \end{center}
\end{titlepage}
\newpage

\section*{Remerciments}
\newpage
\tableofcontents
\printnoidxglossaries
\listoffigures
\newpage

\section{Introduction}
\subsection{L'entreprise}
\paragraph*{}
Orange est à l’origine une entreprise anglaise de télécommunication. Elle a été rachetée par France Télécom en 2000, entreprise française fondée en 1975, devenant par la suite de ce rachat une société internationale. Au 1er juillet 2013, France télécom change de nom est devient Orange, société française qui est alors la 121ème entreprise mondiale avec un chiffre d’affaire de 41 milliards d’euros pour cette même année. Actuellement, Orange emploie 172 000 personnes mondialement, dont 105 000 en France et possède plus de 226 millions de clients dans le monde répartis dans 34 pays dont 11 pays d’Europe. (Voir carte ci-dessous)
\\ \\ IMAGE! \\ \\
\paragraph*{}
Le groupe Orange est majoritairement présent en Europe et Afrique. Il est avant tout un leader de la téléphonie mobile avec un total de 171.6 millions de clients mobile en 2013 au niveau mondial. Mais aussi, leader dans le domaine de l’accès à internet avec 15 millions de clients ADSL et 265 000 clients FTTH, ainsi que la téléphonie fixe avec 42 millions de clients fin 2014 en France. Les pays où le groupe est le plus implanté sont la France, l’Espagne, la Pologne et l’Angleterre. Depuis plusieurs années maintenant Orange essaie de se développer également en Afrique dans le domaine de la téléphonie mobile.
\paragraph*{}
Le secteur d’activité principal du groupe Orange reste les Télécommunications, en étant un opérateur téléphonique majeur en France et dans bien d’autres pays tels que la Pologne, l’Espagne, l’Angleterre, Côte d’Ivoire, Égypte etc. Orange est également un fournisseur d’accès internet et depuis quelques années élargit ses activités à la domotique, vente de contenus cinématographique et musical, médical, applications bancaires et automobiles etc.
\paragraph*{}
Les principaux concurrents d'Orange en France dans le domaine FAI sont principalement Free, Numéricâble, OVH, Nerim, Wifirst et Bouygues Télécoms. Et pour la téléphonie mobile ses principaux concurrents sont SFR, Free et Bouygues Télécom. Tandis qu'au niveau européen sur le domaine téléphonique et FAI, les principaux concurrents sont Deutsche Telekom, Vodafone et O2 en grande majorité.
\paragraph*{}
La branche où j’effectue mon alternance sur 3 ans est OLPS pour Orange Labs Products and Services. Cette branche concerne tous ce qui touche à la recherche et au développement des produits Orange. Anciennement nommé France Télécom R&D, puis rebaptisé OLPS en 2007, cette branche destinée à la recherche de l’ensemble du groupe Orange emploie 3500 personnes exclusivement en France. Fin 2012, le nombre de brevets déposés par Orange Labs s’élevaient à 7493. La R&D est très importante pour Orange qui investit chaque année près de 900 millions d’euros dans ce secteur.

\subsection{Le contexte}
\subsubsection{Le Device Mangement à Orange}
\paragraph*{}
Mon alternance se déroule dans la branche R\&D d’Orange, appelée \gls{ols}. Plus précisément dans l’équipe \gls{care}, qui s’occupe de la gestion des équipements client, c’est-à-dire du « Device Management ».
\paragraph*{}
Le concept de « Device Management » possède plusieurs définitions selon les objets ou équipements gérés, et les équipes qui le mettent en place. Au sens de notre équipe, il est découpé en deux zones détaillées comme suis : 
\begin{itemize}
\subparagraph*{}
\item Le coté client, où l’on retrouve le réseau privé du client, dit le \gls{lan}, avec généralement divers équipements tels que, une passerelle internet, un décodeur TV, un téléphone, une caméra IP, des capteurs domotiques etc.
\item Le coté serveur, se trouvant chez Orange, où l’on va retrouver les serveurs, appelés \gls{acs} qui vont permettre de faire ce que l’on
nomme du Device Management.
\end{itemize}
\paragraph*{}
La communication entre un équipement et son \gls{acs} se fait via le protocole normalisé \gls{cwmp}. Ce protocole respecte le Document TR-069 définie par le \gls{bbf}. Ainsi, chaque équipement et \gls{acs} doit respecter la norme décrite dans le Document TR-069. Il faut donc toujours veiller à ce que les équipements embarquent bien un Client CWMP respectant cette norme. Tous comme les \gls{acs} qui doivent aussi rester à jour de cette norme.
%Import Images
\begin{figure}[!ht]
    \center
    \includegraphics[scale=0.7]{./img/DM-TR-069-screen.png}
    \caption{Réseau de Device Management, côté \gls{acs} et côté Client}
\end{figure}
\paragraph*{}
L’un des objectifs du Device Management, pour l’équipe \gls{care}, est d’apporter un service d’aide et de dépannage aux clients, tous en restant à distance. Dans le but de ne pas avoir à faire déplacer un technicien sur place, pour un problème qui peut être résolu à distance par l’exécution de scripts, lancement de test et analyse, correction de bug. Le rôle de l’équipe \gls{care}, est de concevoir l’intégration de ces outils qui pourront être utilisés à distance.
\paragraph*{}
La supervision et la maintenance du parc Orange sont d’autres activités dans le
périmètre de l’activité du Device Management. Ce parc contient les différents produits
vendus par Orange et qu’Orange s’engage à maintenir. On comprend alors l’importance des
activités de supervision et de maintenance. Pour gérer ce parc, Orange a besoin, entre
autres, d’identifier les différents équipements présents et d’accéder à leurs  caractéristiques. Les outils de Device Management développés au sein de l’équipe \gls{care} permettent, cette fois, de remonter aux \gls{acs} toutes les informations nécessaires pour superviser et maintenir le parc. Il permet également de mettre à jour et corriger des bugs en envoyant de nouvelles versions de firmware aux équipements concernés. \\
\subsubsection{La norme TR-069}
\subsection{Objectifs envisagés}
\subsubsection{Première année}
\subsubsection{Deuxième année}
\subsubsection{Troisième année}

\section{Monté en compétence sur le protocole CWMP}
\subsection{Création d'un ACS Servlet}
\subsection{Études d'équipements}
\subsubsection{Présentation du réseau isolé}
\subsubsection{Test DNS}
\subsubsection{Test de comportement TR-069 d'équipement}
\subsection{Étude de client CWMP}
\subsubsection{Client EasyCWMP}
\subsubsection{Client tr69agent d'Orange}
\subsubsection{Résultats} %conséquence : création du toolkit !
\subsection{Impact sur mon parcours}

\section{Projet principal: Conception et développement d'un outil de test}
\subsection{Contexte}
\subsection{Présentation}
\subsection{Méthode de projet}
\subsection{Travail de préparation}
\subsubsection{Recherche de solution technique}
\subsubsection{Analyse de faisabilité}
\subsection{Conception}
\subsection{Réalisation}
\subsubsection{Travail en équipe}
\subsubsection{Développement}
\subsection{Déploiement}
\subsubsection{Environement}
\subsection{Communication et utilisateur}
\subsection{Livrable du projet}
\subsection{Difficultés, solutions et compétences acquises}
\subsection{Bilan et apport personnel du projet}

\section{Transfert de compétences}

\section{Bilan de compétences}

\section{Conclusion}
\subsection{Atteintes des objectifs}
\subsection{Progression}
\subsection{Synthèse de parcours}

\end{document}
