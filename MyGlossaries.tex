\newglossaryentry{apig}{
    name={API},
    description={is a particular set of rules and specifications that a software 			program can follow to access and make use of the services and resources 			provided by another particular software program that implements that API}
}

\newglossaryentry{tinkg}{
	name={TINK},
	description={Nom du projet réalisé durant ma deuxième et troisième année 				correspondant à l’outil de test d’équipement en ligne du Toolkit TR-069}
}

\newglossaryentry{acsg}{
	name={ACS},
	description={Type de serveur permettant d’administrer et gérer des 						équipements TR-069 que les clients Orange possèdent.}
}

\newglossaryentry{careg}{
	name={CARE},
	description={Équipe dans laquelle j’évolue depuis 3 ans, constituée d’une 				quinzaine de chercheurs/ingénieurs}
}

\newglossaryentry{lang}{
	name={LAN},
	description={Réseau informatique tel que l’on peut le retrouver chez soi}
} 

\newglossaryentry{cwmpg}{
	name={CWMP},
	description={Protocole du \gls{doctr069g}. C’est le protocole de 						communication employé pour que les serveurs \gls{acs} et les équipements 					puissent dialoguer}
} 

\newglossaryentry{bbfg}{
	name={BBF},
	description={Consortium à but non-lucratif se concentrant sur le 						développement de protocole de réseaux télécoms}
} 

\newglossaryentry{doctr069g}{
	name={Document TR-069},
	description={Document édité par le \gls{bbf}. Il contient l’ensemble des 				méthodes et spécification à respecter pour implémenter la norme  \gls{cwmp}}
} 

\newglossaryentry{dhcpg}{
	name={DHCP},
	description={Attribue un paramétrage IP (adresse IP, masque de sous-réseau 				etc) aux machines qui le contactent, afin qu’elles puissent communiquer 			avec d’autres réseaux}
} 

\newglossaryentry{toolkitg}{
	name={Toolkit TR-069},
	description={Solution d’implémentation de Client CWMP d’Orange pour les
		constructeurs d’équipements, comprenant un Client CWMP développé par 				Orange ainsi qu’un outil en ligne pour tester son propre Client CWMP}
} 

\newglossaryentry{dnsg}{
	name={DNS},
	description={Traduit des noms de domaines en l’adresse IP de la machine qui 			porte ce nom}
} 

\newglossaryentry{karmag}{
	name={Karma},
	description={\gls{acs} utilisé et développé par Orange en production}
}

\newglossaryentry{tomcatg}{
	name={Tomcat},
	description={Serveur HTTP gérant les servlets écrit en Java}
}

\newglossaryentry{srcappg}{
	name={Serveur d’application},
	description={Logiciel d’infrastructure offrant un contexte d’exécution pour 			des composants applicatifs}
}

\newglossaryentry{redhatg}{
	name={RedHat Enterprise},
	description={Société multinationale d’édition de distribution Linux}
}

\newglossaryentry{karmabug}{
	name={Karma BU},
	description={Branche internationale de Karma. Instance de Karma destinée aux 			filiales d’Orange autres que la France}
}

\newglossaryentry{farmafrg}{
	name={Karma Fr},
	description={Branche française de Karma. Instance de Karma destinée aux 				clients français.}
}

\newglossaryentry{x69g}{
	name={X69},
	description={Premier module de Karma interrogé par les équipements souhaitant 		communiquer avec Karma. Il permet de rendre compréhensible les trames 				CWMP des équipements clients pour Karma, et inversement de modifier les 			trames créées par Karma pour qu’elles soient compréhensibles pour les 				équipements}
}

\newglossaryentry{datamodelg}{
	name={Data Model},
	description={Modèle de données sous forme d’arbre contenant l’ensemble des
		caractéristiques d’un équipement. Il peut aller de quelques dizaines de 			paramètre à plusieurs milliers selon l’équipement. En modifiant les 				paramètres du data model d’un équipement, on peut modifier son 						fonctionnement. Chaque équipement respectant le Document TR-069 doit 				implémenter un data model, qu’il renvoie à son ACS}
}

\newglossaryentry{kermitg}{
	name={Kermit},
	description={Outil de service cloud de PaaS d’Orange basé sur Open Shift}
}

\newglossaryentry{dockerg}{
	name={Docker},
	description={Logiciel libre automatisant le déploiement d’application dans 				des conteneurs}
}

\newglossaryentry{springg}{
	name={Spring},
	description={Framework open source Java permettant de construire et définir 			l’infrastructure d’une application Java et faciliter son développement et 		ses tests}
}

\newglossaryentry{cxfg}{
	name={CXF},
	description={Framework Open Source Java facilitant le développement de 					service web}
}

\newglossaryentry{hibernateg}{
	name={Hibernate},
	description={Framework Java facilitant la communication avec différents types 		de bases de données}
}

\newglossaryentry{ihmg}{
	name={IHM},
	description={Interface graphique permettant à un utilisateur final de 					communiquer avec l’application}
}

\newglossaryentry{cpeg}{
	name={CPE},
	description={Nom défini par le BBF pour désigner un équipement client}
}

\newglossaryentry{openshiftg}{
	name={Open Shift},
	description={Service Cloud de PaaS proposé par RedHat Enterprise, sur lequel est basé Kermit}
}

\newglossaryentry{linuxg}{
	name={Linux},
	description={Aussi appelé GNU/Linux, cela désigne une suite de systèmes 				d’exploitation, basé sur un noyau Linux}
}

\newglossaryentry{mysqlg}{
	name={MySQL},
	description={Logiciel de gestion de bases de données relationnelles}
}

\newglossaryentry{opensourceg}{
	name={Open Source},
	description={Désigne les logiciels dont la licence permet sa libre 						distribution, un accès au code source pour la création de travaux 					dérivés}
}

\newglossaryentry{userstoriesg}{
	name={User Stories},
	description={Regroupement de fonctionnalités que le logiciel doit embarquer. 			Constitue les Sprint}
}

\newglossaryentry{sprintg}{
	name={Sprint},
	description={Période de développement de quelques semaines au bout de 					laquelle on livre une version potentiellement livrable du produit}
}

\newglossaryentry{releaseg}{
	name={Release},
	description={Version livrable du projet après plusieurs Sprint}
}
\newglossaryentry{agiliteg}{
	name={Agilité},
	description={Méthode de pilotage et de réalisation de projet}
}

\newglossaryentry{trackerg}{
	name={Trackers},
	description={Outil de traçage, permettant de tenir à jour l’avancement et les 		détails d’une tâche}
}

\newglossaryentry{webserviceg}{
	name={Web Services},
	description={Programme exposé sur internet ou intranet permettant la 					communication et les échanges de données entre application de manière 				synchrone ou asynchrone. Dialogue le plus souvent par HTTP}
}

\newglossaryentry{tr069agentg}{
	name={tr069agent},
	description={Client CWMP développé par Orange, écrit en C et proposé dans le 			Toolkit TR-069}
}

\newglossaryentry{frameworkg}{
	name={Framework},
	description={Ensemble de composants logiciels permettant de faciliter la 				construction et la maintenance d’un programme}
}

\newglossaryentry{standupg}{
	name={Stand-up daily meeting},
	description={Réunion quotidienne d’une quinzaine de minutes au maximum faite
		debout où chacun explique où il en est et ce qu’il va faire}
}

\newglossaryentry{comg}{
	name={COM},
	description={Module de Karma permettant d’ordonnancer le traitement des 				requêtes des équipements clients selon leur contenu}
}

\newglossaryentry{firmwareg}{
	name={Firmware},
	description={Ensemble d’instruction et structure de données qui sont intégrées dans un matériel informatique pour qu’il puisse fonctionner}
}

\newglossaryentry{servletg}{
	name={Servlet},
	description={Permet l’extension des fonctions d’un serveur web sur lequel 				elle est déployée. Une servlet est une application web java}
}

\newglossaryentry{productownerg}{
	name={Product Owner},
	description={Rôle de la méthode Scrum, devant établir la liste des 						fonctionnalités du produit attendu et l’ordre dans lequel elles sont 				implémentées}
}

\newglossaryentry{kanbang}{
	name={Kanban},
	description={Méthode agile, consistant à diviser le travail en différentes 				cartes limitées afin de ne pas surcharger les développeurs à un instant 			t}
}

\newglossaryentry{scrumg}{
	name={Scrum},
	description={Méthode agile, consistant à découper le projet en différents 				Sprints qui aboutissent chacun à une démonstration du produit}
}

\newglossaryentry{sgbdg}{
	name={SGBD},
	description={Permet à un utilisateur de manipuler une base de données. Il peut être sous la forme d'un logiciel applicatif, ou bien d'un serveur. Certain peuvent proposer leur propre langage. Parmis les plus connu on peut citer Oracle, MySQL, DB2..}
}